\documentclass[12pt, letterpaper]{article}
\usepackage{amsmath}
\usepackage{amsthm}
\usepackage{amssymb}
\usepackage[parfill]{parskip}
\usepackage{listings}
\usepackage{xcolor}
\usepackage{hyperref}
\usepackage{floatrow}
\usepackage{url}
\usepackage{graphicx}
\usepackage{ragged2e}
\usepackage[normalem]{ulem}

\newtheorem{theorem}{Theorem}
\newtheorem{lemma}{Lemma}
\newtheorem{corollary}[theorem]{Corollary}

\theoremstyle{remark}
\newtheorem*{remark}{Remark}

\lstset{
    frameround=fttt,
    language=[LaTeX]TeX,
    numbers=left,
    breaklines=true,
    delim        = [s][\color{red!50!black}]{$}{$},
    moredelim=[s][\color{red!50!black}]{\\[}{\\]},
    moredelim=[s][\color{red!50!black}]{$$}{$$},
    keywordstyle=\color{blue}\bfseries, 
    %basicstyle=\ttfamily\color{red},
    numberstyle=\tiny\color{gray},
    commentstyle=\color{green!30!black},
    stringstyle = \color{violet}
    }
    
\renewenvironment{boxed}
    {\begin{center}
    \begin{tabular}{|p{0.9\textwidth}|}
    \hline\\
    }
    { 
    \\\\\hline
    \end{tabular} 
    \end{center}
    }

\begin{document}
\title{Introduction to \LaTeX}
\author{Rwitaban Goswami and Mihir Vahanwala}
\maketitle
\tableofcontents
\newpage
\section{Welcome to \LaTeX}

\subsection{\LaTeX? What? Why? How?}
I know what you are thinking. Why do we need these fancy \LaTeX\,documents that need a whole lot of hassle and pain to setup? Couldn't we simply use our trusty old MS-Word, or better still, Notepad? What is \LaTeX\,anyway? Is it a software, or is it a programming language?

Simply put, it's a document preparation system. Well, you may ask, so is MS-Word, what's the difference? 

Well, MS-Word is a WYSIWYG processor (What You See Is What You Get). Why is \LaTeX\, better? Because for many kind of things, what you want to see is very hard to get in MS-Word, but in \LaTeX\, it is deadass easy, you write a single line of code, and you're done!

\LaTeX\, will help you draw diagrams, create table of contents, and of course, write mathematical equations, which is what its most known for among non \LaTeX\, people ;)

But the most important thing \LaTeX\, does is it separates the content from the presentation. The content often goes between tags, and the presentation of that content is defined by the properties of the tags. Sounds familiar? Yes, Just like HTML, \LaTeX\, is also a markup language, so it is terribly easy to understand.

\subsection{Motivation}
\LaTeX{} (pronounced lay-tech or laa-tech, and definitely not lay-teks which will make others think you have some kind of a weird fetish) is a free, powerful, and absolutely indispensable markup tool to typeset elegant technical documents. \LaTeX{} allows us to write complex mathematical equations without much fuss; its environments save us the hassle of organising large documents manually; with \LaTeX{} we can showcase code and render almost any scientific illustration. \LaTeX{} is paradise for anyone who works in STEM. Once you have experience, you can typeset assignments, papers, articles and theses with unprecedented ease, using \LaTeX.

This popular graph from Marko Pinteric is indeed apt:
\begin{figure}[H]
\centering
\includegraphics[width=0.6\linewidth]{ease-graph.png}
\end{figure}

\subsection{Deciding you \LaTeX{} workflow}
Now if you want to work with \LaTeX, you have to decide your \LaTeX\, workflow. What do we mean by that?

Well, by itself, \LaTeX\, is just a markup language. It just defines what your text should look like in your final document. If you write a .tex file (yes, that is the extension of \LaTeX\, files), it is not going to convert itself to the final document (typically a pdf)

So you need some software to do that conversion for you. And as all software go, you can have your software either in GUI (normal software with a graphical interface) or CLI (like a PowerShell or a Bash terminal. Don't worry if you don't recognize these terms, it is not necessary for \LaTeX)

\textit{If you are looking to work in development, or think you'll be finding yourself writing any sort of code in the future, I highly recommend switching to a CLI workflow for} \textbf{all} \textit{ your work} \textbf{right now.}

We have listed the possible workflows in order of the familiarity with code writing or \LaTeX{} skillset required, so pick and choose as you see fit:

\begin{enumerate}
    \item Overleaf
        \begin{itemize}
            \item Difficulty level: \includegraphics[width=0.1\textwidth]{Easy.png}
            \item Flow: GUI
            \item OS: Any, use on web browser
            \item Setup time: None
            \item Description: If you don't want to install any software, and have no experience in writing code before, and don't plan on writing huge documents and/or environments of yourself, you can go for this
            \item How to Use: Just log in on Overleaf.com, type your \LaTeX{} code on the right hand and click 'Recompile' or hit CTRL+S to see your pdf appear magically on the left side
            \begin{figure}[h]
                \centering
                \includegraphics[width=0.8\textwidth]{Overleaf.png}
                \caption{Overleaf}
                \label{fig:Overleaf}
            \end{figure}
            
        \end{itemize}
        \item TeXStudio \label{it:texstudio}
        \begin{itemize}
            \item Difficulty level: \includegraphics[width=0.1\textwidth]{Medium.png}
            \item Flow: GUI
            \item OS: Windows/Linux/OSX
            \item Setup time: As long as it takes to install
            \item Description: The wonder of this is that it provides an intuitive GUI for you to use, but the \LaTeX{} distribution installation itself is separate from the GUI. So all you need to do is install this, and it takes care of the \LaTeX{} distribution by itself. If you find the underlying distribution to be lacking in some packages, then it may be a bit of some pain to first find out which distribution TeXStudio is using behind the curtains, and then install packages for that. This will work nicely if you want to do the \ref{it:texlive}$^{\text{th}}$ option but with a GUI instead of the text editor.
            \item How to Use: You have to install the setup according to your OS from \url{https://www.texstudio.org/#download}, and install the application. It provides a similar intuitive GUI to overleaf
             \begin{figure}[h]
                \centering
                \includegraphics[width=0.8\textwidth]{TexStudio.png}
                \caption{TexStudio}
                \label{fig:TexStudio}
            \end{figure}
        \end{itemize}
        
        \item TeXWorks \label{it:texworks}
        \begin{itemize}
            \item Difficulty level: \includegraphics[width=0.1\textwidth]{Medium.png}
            \item Flow: GUI
            \item OS: Windows/Linux/OSX
            \item Setup time: As long as it takes to install
            \item Description: This works pretty similarly to \ref{it:texstudio}. TeXStudio, in that it provides an intuitive GUI to use independent of the \LaTeX{} distribution. It even lets you choose the \LaTeX{} distribution you want, and you can setup packages for it accordingly.
            \item How to Use: You have to install the setup according to your OS from \url{http://www.tug.org/texworks/}, and install the application. It will provide a window for you to write your \LaTeX{} code in, and when you compile it, it produces the pdf in a \textit{separate} window
             \begin{figure}[h]
                \centering
                \includegraphics[width=0.8\textwidth]{TexWorks.png}
                \caption{TexWorks}
                \label{fig:TexWorks}
            \end{figure}
        \end{itemize}
        
        \item TeXLive + GUI Text Editor \label{it:texlive}
        \begin{itemize}
            \item Difficulty level:  \includegraphics[width=0.1\textwidth]{Hard.png}
            \item Flow: CLI + Text Editor
            \item OS: Windows/Linux/OSX
            \item Setup time: As long as it takes to understand your OS's CLI, and a hell lot of time for all of TexLive packages to install
            \item Description: Technically, TexLive, is one of the \LaTeX{} distributions that work under the hood for \ref{it:texstudio}. TeXStudio or \ref{it:texworks}. TeXWorks. Instead of using a GUI for utilizing the distribution, you are going to write your \LaTeX{} code in a editor itself, and call the CLI command to compile your pdf file yourself.
            \item How to Use: Steps to get the TeXLive distribution depend on your OS
                \begin{enumerate}
                    \item Linux: Open up your terminal and execute \\ \lstinline[language=bash]{sudo apt-get install texlive # to get the most common packages (not including BiBLaTeX)} \\ \textbf{or} \\ \lstinline[language=bash]{ sudo apt-get install texlive-full # to get all of the packages, but the install time is high} \\
                    Now open up your favorite text editor (some non exhaustive options: Notepad++, Sublime Text, VSCode) and write your \LaTeX{} code. Open up a terminal side by side. Also open the resulting pdf side by side. Whenever you want to see your pdf update, execute \\ \lstinline[language=bash]{pdflatex /path/to/tex/file} \\ and your pdf file will be created in the same directory
                    \item Windows: Install TexLive from \url{https://www.tug.org/texlive/acquire-netinstall.html}. The rest of the instructions are the same as for Linux, except that you open a \textit{cmd terminal} or a \textit{powershell window} instead of the Linux terminal. Also, Windows may not automatically update your pdf once you execute
                    \\ \lstinline[language=bash]{pdflatex /path/to/tex/file} \\
                    so you may need to keep refreshing your pdf if your reader is a browser. Windows will also definitely now allow pdflatex to execute if your pdf is open in a reader like Acrobat
                    \item OSX: If you have homebrew, you can open your terminal and install the distribution via \\
                    \lstinline[language=bash]{brew cask install mactex} \\
                    This will also install the TexWorks GUI along with it, so you could also use that. Otherwise it is pretty similar to working with Linux
                \end{enumerate}
            \begin{figure}[h]
                \centering
                \includegraphics[width=0.8\textwidth]{Sublime.png}
                \caption{Sublime Text + TeXLive}
                \label{fig:Sublime}
            \end{figure}
        \end{itemize}
        ~
        \newline
        \item TeXLive + VSCode + \LaTeX{} workshop plugin
        \begin{itemize}
            \item Difficulty level: \includegraphics[width=0.1\textwidth]{Hard.png}
            \item Flow: VSCode CLI + VSCode Editor + VSCode plugin (to view pdf)
            \item OS: Windows/Linux/OSX
            \item Setup time: Takes time and care to load plugin and all its dependancies
            \item Description: Does everything, from the CLI, to the text editor, to the pdf and its updation, from within VSCode itself
            \item How to Use: First install VSCode from \url{https://code.visualstudio.com/download}. Then install the \LaTeX{} workshop plugin on VSCode by James Yu. Install TeXlive as from before. Install latexmk from \url{https://ctan.org/pkg/latexmk}. Once you finish setting up your VSCode settings, you are good to go. The plugin will automatically refresh the pdf for you as soon as the tex file is saved. It even does things like compile the table of contents, the bibliography, and references in one go. It will even take your cursor from the point in the code to the point in the pdf! The only hassle is the installation
            \begin{figure}[h]
                \centering
                \includegraphics[width=0.8\textwidth]{VSCode.png}
                \caption{VSCode + TeXLive + Plugin}
                \label{fig:VSCode}
            \end{figure}
        \end{itemize}
        
        \item TeXLive + CLI Text Editor
        \begin{itemize}
            \item Difficulty level: \includegraphics[width=0.1\textwidth]{Impossible.png}
            \item Flow: CLI + CLI Editor
            \item OS: Linux/OSX (Windows but with WSL only)
            \item Setup time: Very steep learning curve to learn CLI text editors
            \item Description: Same as \ref{it:texlive}., but the text editor is now CLI. Do this if you are extremely familiar and regularly use a CLI text editor (like Vim, Nano, Emacs etc), or want to start learning it.
            \item How to Use: Setup of TeXLive is same as in \ref{it:texlive}. The CLI editor can also be installed through CLI. Usage is same as before, except that have only \textit{two} windows open side by side, one with the pdf open and one with your text editor (which should have its own terminal inside it). You can also use this approach with Windows, but you can only get a CLI editor (like Vim), through WSL (Windows Subsystem for Linux), which basically runs a command line version of Linux on your Windows (not like an emulator or a virtual machine)
            \begin{figure}[h]
                \centering
                \includegraphics[width=0.8\textwidth]{Vim.png}
                \caption{Vim + TeXLive}
                \label{fig:vim}
            \end{figure}
        \end{itemize}
\end{enumerate}

\subsection{Writing your first \LaTeX{} document}
\LaTeX{} is commonly used to typeset PDFs, and the typesetting code is written in a \verb!.tex! file. Make a file, say, \verb!first.tex!. 

Type in the following code and then hit typeset (or use whatever flow you selected in the previous section):
\begin{lstlisting}
\documentclass[12pt, letterpaper]{article}
%this tells LaTeX that the document is to be treated as an article, and activates features accordingly. Optional information about the font size and type of paper is also supplied.

%the percentage sign is used to make comments! This will not be read by the compiler!

%This part of the .tex file, before you declare \begin{document}, is called the PREAMBLE.

\begin{document}
\title{My First Document}
\author{John Doe}

%if you do not write the next line, the compile date will be typeset automatically
\date{June 14, 2020}

\maketitle

Hello World!
%This is a quiet comment.

\end{document}
\end{lstlisting}

Look, we did it ourselves, and here's what we got. 
\begin{figure}[H]
\centering
\includegraphics[width=0.85\linewidth]{first.png}
\end{figure}
Try it!

\section{\LaTeX{} Basics}
\subsection{The concept of environments}
Observe the source code of your first document. It has a preamble, and a body that is contained between \verb!\begin{document}! and \verb!\end{document}!. Every document has this structure.

The pair \verb!\begin{document}! and \verb!\end{document}! are said to define the \emph{document environment.} All \LaTeX{} files have the document environment, and the \verb!\documentclass{}! in their preamble, but that is not the only environment that is. Environments are used to format blocks of text in \LaTeX{} documents. Environments are delimited by the opening tag \verb!\begin! and the closing tag \verb!\end! Everything inside those tags will be formatted specially, depending on the environment. Environments can be nested: of course, as a trivial example, all the environments we will use are contained in the document environment. 

In this tutorial itself, we will see simple applications of environments to center-align text, and also to create lists. Later on in the course, the usage of environments will be much more involved. 

\subsection{Playing around with text}
\begin{boxed}
This is plain text. 
\textbf{This is bold text}. 
\textit{This is italicised text.}
\emph{Another way.}
\textsc{This is capitalised text.} 
\underline{This is underlined text.} 
\textbf{\textit{This is bold text. Oh wait, it's italicised as well.}} \textcolor{blue}{This is blue text.}
\color{violet}
I made a command to turn the default colour violet for the rest of this demo. 
\textcolor{black!70}{Or do you prefer a touch of grey?}
\textcolor{black!30}{I have fifty shades, actually.}
\textcolor{black}{Darker.}
\textcolor[wave]{600}{This colour is what light of wavelength 600 nm looks like.}
\textcolor{blue!40!green!55!black}{This turquoise is prepared by mixing blue, green and a hint of black.}
\textcolor{-green}{This colour is the complement of green. I wonder if it has a name more specific than pink.}
Back to violet, it's default.
\tiny{This is tiny text.}
\small {This is small text.}
This is regular sized text.
\large{This is large text.}
\huge{This is huge text.}
\end{boxed}
In order to get these shades, we used the \verb!xcolor! package. To make the code below work, type \verb!\usepackage{xcolor}! in the preamble, i.e. before the declaration \verb!\begin{document}! 

Here's how we did it.
\begin{lstlisting}
This is plain text. 
\textbf{This is bold text}. 
\textit{This is italicised text.} 
\emph{Another way.}
\textsc{This is capitalised text.} 
\underline{This is underlined text.} 
\textbf{\textit{This is bold text. Oh wait, it's italicised as well.}} \textcolor{blue}{This is blue text.}
\color{violet}
I made a command to turn the default colour violet for the rest of this demo. 
\textcolor{black!70}{Or do you prefer a touch of grey?}
\textcolor{black!30}{I have fifty shades, actually.}
\textcolor{black}{Darker.}
\textcolor[wave]{600}{This colour is what light of wavelength 600 nm looks like.}
\textcolor{blue!40!green!55!black}{This turquoise is prepared by mixing blue, green and a hint of black.}
\textcolor{-green}{This colour is the complement of green. I wonder if it has a name more specific than pink.}
Back to violet, it's default.
\tiny{This is tiny text.}
\small {This is small text.}
This is regular sized text.
\large{This is large text.}
\huge{This is huge text.}
\end{lstlisting}

Play around with colours (or colors, if you prefer American), and show us a glimpse your favourite shade in the assignments!

\subsection{Paragraphs and Formatting}
We recommend using the \verb!ragged2e! package to toggle between justifications. Type \verb!\usepackage{ragged2e}! in the preamble.

\begin{boxed}
\begin{center}
    The set of paragraphs in this box are Center Aligned using the center environment. To finish this paragraph and start a new one, you could use the enter key twice.
    
    Like so. Another way to tell the `compiler' that you'd like to typeset a new line, is to use double backslash. \\
    See? It worked. There's yet another command to start a new paragraph, as you will see in the code. \par
    There we go. Observe the code. We have a way of introducing vertical space between paragraphs.
    
    \vspace{1.5em}
    
    Like this. \hspace{2.5em} We can also introduce horizontal space.
\end{center}
\end{boxed}

Here is the complete source code:

\begin{lstlisting}
\begin{center}
    The set of paragraphs in this box are Center Aligned using the center environment. To finish this paragraph and start a new one, you could use the enter key twice.
    
    Like so. Another way to tell the `compiler' that you'd like to typeset a new line, is to use double backslash. \\
    See? It worked. There's yet another command to start a new paragraph, as you will see in the code. \par
    There we go. Observe the code. We have a way of introducing vertical space between paragraphs.
    
    \vspace{1.5em}
    
    Like this. \hspace{2.5em} We can also introduce horizontal space.
\end{center}
\end{lstlisting}

\raggedright 
Before beginning this paragraph, we declared the command \verb!\raggedright!, and now, as you can see, our text is left justified. We will now add a few inconsequential lines, just to make the paragraph bigger, and the effect more pronounced.

We will now add a few inconsequential lines, just to make the paragraph bigger, and the effect more pronounced. The left justification is here to stay, until we do something about it. Should we? Well, there's a lesson to teach, so...

\raggedleft
Before beginning this paragraph, we declared the command \verb!\raggedleft!, and now, as you can see, our text is right justified. We will now add a few inconsequential lines, just to make the paragraph bigger, and the effect more pronounced.

We will now add a few inconsequential lines, just to make the paragraph bigger, and the effect more pronounced. The right justification is here to stay, until we do something about it. Should we? Well, there's a lesson to teach, so...

\justifying
Finally, we declared the command \verb!\justifying! right before this paragraph so our text is justified as before, and all is good. We will now add a few inconsequential lines, just to make the paragraph bigger, and the effect more pronounced.

\setlength{\parindent}{7em}
The beginning of this paragraph has a lot of indentation. This is because we issued the command \verb!\setlength{\parindent}{7em}! right before it.

The super indentation is here to stay until we tinker with it again.

\setlength{\parindent}{2em}
This is reminiscent of the time we'd use our fingers to decide indentation in our notebooks.

\setlength{\parindent}{0em} 
And this returns things to the way they were before.

All we did this time was declare \verb!\setlength{\parindent}{0em}!. If you want to switch off indentation for a particular paragraph, use \verb!\noindent! right before it.

This document has \verb!\usepackage[parfill]{parskip}! declared in its preamble. This is convenient to have vertical separation between paragraphs: a series of empty lines in the source code is typeset as an empty line in the PDF. And, for some reason, Mihir does not like the beginning of paragraphs to be indented. Make sure you import this \emph{before} \verb!ragged2e!.

We will show you the source code for the next few lines. Honestly, we have no idea why it sounds like `clickbait'.

Rwitaban and Mihir wondered what's the best way to format paragraphs, till they came across this package.




There's enough space in the source to fit a truck, but they'll have no idea when we typeset it.

\begin{lstlisting}
We will show you the source code for the next few lines. Honestly, we have no idea why it sounds like `clickbait'.

Rwitaban and Mihir wondered what's the best way to format paragraphs, till they came across this package.




There's enough space in the source to fit a truck, but they'll have no idea when we typeset it.
\end{lstlisting}

A question that must've popped in your mind: What exactly does \verb!2.5em! mean in a command that deals with length? \verb!em! is a unit of length that is roughly equal to the width of an uppercase `M' in the current font. The math unit \verb!mu! is related: \verb!em! = 18 \verb!mu!, where \verb!em! is from the math symbols family.

Another unit defined this way is \verb!ex!: roughly the height of a lowercase `x' in the current font.

Then of course, there are more standard units like \verb!pt!, \verb!cm!, \verb!mm! and \verb!in! (which stands for inch).

The most commonly used lengths include \verb!\linewidth! (width of the line in the current environment), \verb!\parskip! (vertical space between paragraphs), \verb!\topmargin! (length of the top margin). \href{https://www.overleaf.com/learn/latex/Lengths_in_LaTeX}{The linked Overleaf tutorial} is a helpful guide.

\subsection{Lists}
\begin{boxed}
\begin{itemize}
    \item This is a list.
    \item It is an unordered list.
    \item It is created using the `itemize' environment.
    \begin{itemize}
        \item Environments can be nested.
        \item So can unordered lists.
    \end{itemize}
    \item Look at the code below to see how it's done.
    %This is the code. 
\end{itemize}
\end{boxed}

\begin{lstlisting}
\begin{itemize}
    \item This is a list.
    \item It is an unordered list.
    \item It is created using the `itemize' environment.
    \begin{itemize}
        \item Environments can be nested.
        \item So can unordered lists.
    \end{itemize}
    \item Look at the code below to see how it's done.
    %This is the code.
\end{itemize}
\end{lstlisting}

\begin{boxed}
\begin{enumerate}
    \item This is an ordered list. 
    \item The only difference in the code is, \verb!itemize! is replaced by \verb!enumerate!.
    \begin{enumerate}
        \item Nesting
        \begin{enumerate}
            \item More nesting
            \begin{enumerate}
                \item Man stop.
                \item This is the deepest nesting that is supported.
            \end{enumerate}
        \end{enumerate}
    \end{enumerate}
\end{enumerate}
\end{boxed}

\LaTeX{} automatically renders nested lists, both ordered and unordered, in distinct styles. Changing these styles is, in our opinion, a bit too finicky, and beyond the scope of this basic tutorial. You will learn more about it, once you gain mastery over environments and commands.

Here's a way to make descriptive lists:
\begin{boxed}
\begin{description}
\item [\textbf{MA 105}] Calculus
\item [\textbf{MA 106}] Linear Algebra
\item [\textbf{MA 108}] Differential Equations
\end{description}
\end{boxed}
The code is simple enough:
\begin{lstlisting}
\begin{description}
\item [\textbf{MA 105}] Calculus
\item [\textbf{MA 106}] Linear Algebra
\item [\textbf{MA 108}] Differential Equations
\end{description}
\end{lstlisting}

\section{Resources}
This, and subsequent tutorials are written with the intention to motivate and help you get started. We recommend you follow the tutorials and try out what we demonstrate yourself. The majority of learning will take place on the job, that is, when you type in code yourself. Here are pointers to some helpful resources; they even helped us \sout{blatantly copy} design this material.
\begin{itemize}
    \item \url{https://www.overleaf.com/learn/latex/Tutorials}
    \item \url{https://www.latex-tutorial.com/tutorials/}
\end{itemize}
These sources are vast, and will prove useful whenever you sit down to write \LaTeX. Also, if you run into a wall, a simple Google search is an underrated trick; you'll often find a helpful TeX StackExchange thread (or any similar forum) in the first few results. Moreover, we encourage you to follow the links in our tutorials, and check out the citations, if any.
\end{document}

